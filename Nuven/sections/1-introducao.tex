\chapter{Introdução}\label{chapter:introducao}

Segundo \cite{solomon}, "o exame de Papanicolaou, também conhecido como citologia cervical ou esfregaço de Papanicolaou, é um teste de rastreamento amplamente utilizado para a detecção precoce de câncer cervical e lesões pré-cancerosas, sendo essencial para a saúde pública, pois permite a identificação de alterações celulares que podem evoluir para o câncer, possibilitando intervenções precoces e reduzindo significativamente a mortalidade associada ao câncer cervical". Dentro da área de visão computacional, as redes neurais convolucionais são uma ferramenta poderosa para a extração e classificação de atributos em imagens, incluindo as obtidas no exame de Papanicolaou, sendo sua aplicação extremamente útil no auxílio ao diagnóstico, uma vez que essas redes são capazes de analisar grandes volumes de imagens de forma rápida e precisa. As redes neurais convolucionais podem extrair automaticamente características relevantes das imagens de células cervicais, como anomalias na textura, forma e padrão celular, que são indicativos de possíveis lesões pré-cancerosas ou câncer cervical.

No contexto das redes neurais convolucionais, a sua utilização pode melhorar a precisão dos diagnósticos, reduzindo a taxa de falsos positivos e falsos negativos, e permitindo uma triagem mais eficiente e consistente. Além disso, a automação do processo de análise de imagens com redes neurais convolucionais pode aliviar a carga de trabalho dos citologistas, permitindo que eles se concentrem em casos mais complexos e que requerem uma avaliação mais detalhada. Esse processo de análise é realizado automaticamente através das camadas convolucionais das, onde filtros são aplicados às imagens para extrair características de maneira hierárquica e progressivamente mais complexa. O processo é realizado automaticamente através das camadas convolucionais, onde filtros são aplicados à imagem para extrair essas características de maneira hierárquica e progressivamente mais complexa. \cite{forsyth} destacam a importância da extração de atributos em seu livro seminal sobre visão computacional, descrevendo-a como "o processo de transformar uma imagem em uma representação mais compacta que é mais fácil de analisar". As redes neurais convolucionais realizam essa transformação de maneira eficiente, criando uma representação compacta composta por atributos relevantes, o que é essencial para reduzir a complexidade computacional e facilitar a interpretação automática de imagens.

As camadas densamente conectadas que recebem como entrada a representação compacta extraída pelas camadas convolucionais. Esses classificadores atribuem rótulos ou categorias a objetos, ou regiões de interesse na imagem, analisando as características extraídas para determinar a classe à qual cada objeto pertence. \cite{szeliski} afirma que "os classificadores são uma ferramenta essencial em visão computacional, permitindo a automação de tarefas como reconhecimento de objetos e detecção de padrões em imagens." No caso das redes neurais convolucionais, esses classificadores são integrados no final da rede neural, tornando-a uma solução completa e eficiente para a análise de grandes conjuntos de dados visuais, automatizando tarefas de reconhecimento e categorização.
