\chapter{Conclusões}\label{chapter:conclusoes}

A conclusão deste estudo evidencia a eficácia e a robustez do modelo da rede neural convolucional desenvolvida na classificação de imagens de células citológicas, apresentando alto desempenho em termos de acurácia e outras métricas-chave. Estes resultados são promissores para a aplicação clínica na detecção precoce de anormalidades celulares. Futuros trabalhos podem explorar arquiteturas de modelos mais complexas e conjuntos de dados maiores para melhorar ainda mais a precisão e a confiabilidade do sistema. O presente estudo investigou a eficácia do modelo de rede neural convolucional com a função de ativação ReLU6 para a classificação de imagens de células citológicas, diferenciando entre células normais e anormais. A análise dos resultados demonstrou um desempenho robusto do modelo, com métricas de avaliação que indicam uma alta capacidade de generalização e precisão.

Os indicadores mostrados no tópico acima afirmam que o modelo possui uma alta taxa de acertos tanto para a classe normal quanto para a classe anormal, com um número relativamente baixo de erros de classificação. Os resultados deste estudo são encorajadores e demonstram o potencial das redes neurais convolucionais com ReLU6 para a classificação automática de imagens citológicas. As implicações clínicas são significativas, pois um sistema automatizado eficiente pode contribuir para a detecção precoce de anormalidades celulares, melhorando as taxas de sucesso no tratamento de câncer cervical. Para trabalhos futuros, recomenda-se explorar arquiteturas de redes neurais mais complexas e conjuntos de dados maiores para aumentar ainda mais a precisão e a robustez do modelo. Além disso, a implementação de técnicas de data augmentation e ajuste de hiper parâmetros pode ajudar a melhorar a variabilidade, a especificidade e o desempenho geral do sistema.