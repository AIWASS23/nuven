\resizebox{!}{0.3cm}{\textbf{RESUMO}}\\

Este estudo apresenta um processo detalhado para a classificação de imagens de células utilizando redes neurais convolucionais, com foco na distinção entre células normais e anormais em imagens digitais de exames de Papanicolaou. O objetivo é avaliar a eficácia e eficiência do modelo proposto na tarefa de classificação de imagens celulares, diferenciando células normais de anormais com base em imagens digitais de microscopia, provenientes de exames de Papanicolaou, visando aprimorar a precisão dos diagnósticos médicos e reduzir a carga de trabalho dos citologistas. O principal propósito é avaliar a eficácia e eficiência do modelo proposto na tarefa de classificação de tais imagens.

As imagens de células provenientes de exames de Papanicolaou foram categorizadas em duas classes principais: \textbf{Normal} e \textbf{Anormal}, sendo as imagens organizadas meticulosamente em um \textbf{DataFrame} contendo as colunas \textbf{filename} e \textbf{class}, facilitando assim a referência e manipulação durante o processamento. Para garantir a integridade dos dados, a detecção e eliminação de arquivos duplicados foram realizadas de forma eficiente utilizando \textbf{hashes MD5}. No processo de carregamento e pré-processamento das imagens, cada imagem foi redimensionada e posteriormente normalizada. Essa etapa foi crucial para preparar as imagens de forma adequada para a entrada no modelo de redes neurais convolucionais. Além disso, os rótulos associados a cada imagem foram convertidos para o formato one-hot encoding, uma técnica essencial que simplifica o treinamento e a classificação. O modelo foi então construído utilizando a biblioteca TensorFlow em conjunto com Keras, seguindo uma arquitetura sequencial. A configuração incluiu várias camadas convolucionais para capturar características significativas das imagens, camadas de max pooling para redução de dimensionalidade, e camadas densamente conectadas para processamento final. A função de ativação ReLU6 foi escolhida para promover a não linearidade na extração de características, enquanto camadas de dropout foram inseridas para mitigar o overfitting durante o treinamento. Para otimização do modelo, a função de perda de entropia cruzada categórica foi utilizada, e o algoritmo Adam foi empregado como otimizador. O modelo foi treinado extensivamente com os dados de treino e validado com os dados de teste para garantir robustez e precisão na classificação das imagens de células. Essa abordagem integrada e meticulosa demonstra o potencial das redes neurais convolucionais na análise de imagens médicas, especificamente no contexto do diagnóstico precoce de anomalias celulares identificadas em exames de Papanicolaou.